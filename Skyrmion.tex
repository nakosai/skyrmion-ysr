\documentclass[twocolumn,showpacs,floatfix,nofootinbib,longbibliography]{revtex4-1}
\usepackage{graphicx}
\usepackage{bm} % bold math
\usepackage{amssymb} % use this package to enable \nrightarrow command
\usepackage{amsmath} % use this package to enable \xrightarrow command
\usepackage{braket} % use for Dirac bra-kets : \rangle \labgle & \mid
\usepackage{natbib} % bibtex package
\usepackage{hyperref}


\begin{document}

\title{Currents induced by magnetic impurities in superconductors with spin-orbit coupling}

\author{Sho Nakosai$^{1,2}$}
\author{Sergey S. Pershoguba$^{2}$}
\author{Alexander V. Balatsky$^{,3}$}
\affiliation{$^1$Department of Applied Physics, University of Tokyo, Tokyo 113-8656, Japan}
\affiliation{$^2$Nordita, Center for Quantum Materials, KTH Royal Institute of Technology, and Stockholm University, Roslagstullsbacken 23, S-106 91 Stockholm, Sweden}
\affiliation{$^3$Institute for Materials Science, Los Alamos National Laboratory, Los Alamos, NM 87545, USA}

\date{\today}

\begin{abstract}
Skyrmions are nice
\end{abstract}

\pacs{ }   

%%%%%%%%%%%%%%%%%%%%%%%%%%%%%%%%%%%%%%%%%%%%%%%%%%%%%%%%%%%%%%%%%%%%%%%%%%

\maketitle
%%%%%%%%%%%%%%%%%%%%%%%%%%%%%%%%%%%%%%%%%%%%%%%%%%%%%%%%%%%%%%%%%%%%%%%%%%%
\section{Introduction} \label{sec:intro}
%%%%%%%%%%%%%%%%%%%%%%%%%%%%%%%%%%%%%%%%%%%%%%%%%%%%%%%%%%%%%%%%%%%%%%%%%%%

General context of skyrmions asd tolopolical excitations: memory, manipulation, local creation via SP STM.
Extension of skyrmion discussion to the case of hybrid structures: SC and Skyrmion. What are the consequenc of brining topological exchange field into SC. Question we address is the possible local spectroscopic signatures of SC quasiparticles in SC due to skyrmion field. We know from the past discussion that there are impurity bound states in SC near magnetic impurities. We have now the framework to address formation of bound states. Talk about local single impurity limit (YSR) and show the cartoon of the local and extended skyrmion and spectra. There are two effects: local scattering and Zeeman field hence the DOS will be split etc.  Draw similarities and differences with single imp.
In parallel with skyrmion discovery the local imaging using magnetic probes like MFM and SP-STM allowed one to image the matter at atomic resolution while also resolving spin content of electron carriers in the substrate.  
Here we prove the existence of the new type of localized excitation on the skyrmion core we call  Sc-YSR state (alternative is skyrmion bound state (sbs)).  Show the main results upfront in the introduction. Both LDOS and SP-LDOS. 
Main section: 

Introduce T matrix and results for analytic solution.  
Introduce the numerical approach and presenst the results oas a function of position and as a function of energy. Kind of same figs as in Sho’s talk. 
Discuss the results and what it means, how big the signal is etc. Unfortunately we do not see any topological state at zero energy and as such these result represent a new kind of magnetic texture induced states that exhibit intragap states.


%%%%%%%%%%%%%%%%%%%%%%%%%%%%%%%%%%%%%%%%%%%%%%%%%%%%%%%%%%%%%%%%%%%%%%%%%%%
\section{Skyrmions in ferromagnetic films} \label{sec:skyrmion}
%%%%%%%%%%%%%%%%%%%%%%%%%%%%%%%%%%%%%%%%%%%%%%%%%%%%%%%%%%%%%%%%%%%%%%%%%%
Define skyrmions.  Write topological number. Define magnetic moments of the skyrmion (anapole,monopole etc). Discuss different types of skyrmions and that they are equivalent for electrons. 

Added more info


%%%%%%%%%%%%%%%%%%%%%%%%%%%%%%%%%%%%%%%%%%%%%%%%%%%%%%%%%%%%%%%%%%%%%%%%%%%
\section{T-matrix analysis} \label{sec:analytics}
%%%%%%%%%%%%%%%%%%%%%%%%%%%%%%%%%%%%%%%%%%%%%%%%%%%%%%%%%%%%%%%%%%%%%%%%%%%
Superconductor-ferromagnet heterostructures were recently proposed as a viable platform for realizing topological superconductivity (TS) \cite{Lutchyn2010,Oreg2010, Sau2010}, which can host Majorana fermion quasiparticles at vortex cores and boundaries \cite{Kitaev2001, Alicea, Beenakker2013}. Majorana fermions obey non-Abelian statistics and may be utilized for topological quantum computation \cite{Read2000, Ivanov2001, Nayak2008}.  The key ingredients driving these systems in the topologically non-trivial regime are the spin-orbit coupling (SOC) and magnetism. Recently, the search for experimental realizations of TS has also led to engineering the impurity bands of the Yu-Shiba-Rusinov (YSR) states \cite{Yu,Shiba,Rusinov}, induced by magnetic atoms on the surface of a superconductor \cite{Choy2011, Nadj-Perge2013, Klinovaja2013, Vazifeh2013, Braunecker2013, Pientka2013, Nakosai2013, Poyhonen2014, Reis2014, Brydon2015, Rontynen2014, Li2015}. Following this recipe, zero-energy peaks in the tunneling spectrum were recently measured at the ends of a one-dimensional (1D) chain of magnetic atoms \cite{Yazdani2014}. Such a tunneling spectrum could be the evidence of Majorana edge states, although alternative explanations are also possible \cite{Sau2015}.

The interplay of SOC and magnetism has another remarkable consequence. Consider a two-dimensional (2D) surface of a 3D superconductor. The effective Hamiltonian of the surface $h(\bm p) = \frac{{\bm p}^2}{2m} + \lambda\left( \bm\sigma\times\bm p\right)_z$ contains Rashba SOC due to the absence of inversion symmetry at the surface. Then, the velocity operator $\bm v= {\frac{dh(\bm p)}{d\bm p} =\frac{\bm p}{m}+ \lambda \,\hat{\bm z}\times\bm\sigma}$ contains a spin-dependent term that gives an extra contribution to the current
\begin{equation}
	\bm j_{\rm extra} = \lambda\,  \hat{\bm z}\times\langle\bm\sigma\rangle . \label{main}
\end{equation}
A ferromagnet proximity-coupled to the superconductor would render a finite spin polarization $\langle \bm \sigma \rangle \neq 0$ and thus generate a current as schematically shown in Fig. The phenomenon of driving a current with magnetism is known as the magnetoelectric effect. This effect may vanish in metals due to dissipation but survives in superconductors lacking inversion symmetry \cite{Levitov1985, Edelstein1989, Edelstein1995, Yip2001, BauerSigrist2012}. The magnetoelectric effect was also recently discussed in a pure 1D model of TS \cite{Ojanen2012}. 

t-matrix is good too.

very very good.

very very very good.


\begin{equation}
 T = \frac{V}{1-vg_{00}}
\end{equation}

%%%%%%%%%%%%%%%%%%%%%%%%%%%%%%%%%%%%%%%%%%%%%%%%%%%%%%%%%%%%%%%%%%%%%%%%%%%
\section{Numerical analysis} \label{sec:numerics}
%%%%%%%%%%%%%%%%%%%%%%%%%%%%%%%%%%%%%%%%%%%%%%%%%%%%%%%%%%%%%%%%%%%%%%%%%%%
Text added by Sho.
2nd edition. 
Another addition. 
Further addition.
%%%%%%%%%%%%%%%%%%%%%%%%%%%%%%%%%%%%%%%%%%%%%%%%%%%%%%%%%%%%%%%%%%%%%%%%%%%
\section{Conclusion} \label{sec:conclusion}
%%%%%%%%%%%%%%%%%%%%%%%%%%%%%%%%%%%%%%%%%%%%%%%%%%%%%%%%%%%%%%%%%%%%%%%%%%%




\newpage
%%%%%%%%%%%%%%%%%%%%%%%%%%%%%%%%%%%%%%%%%%%%%%%%%%%%%%%%%%%%%%%%%%%%%%%%%%%%%
%\bibliographystyle{apsrev4-1}
\bibliography{Skyrmion}
%%%%%%%%%%%%%%%%%%%%%%%%%%%%%%%%%%%%%%%%%%%%%%%%%%%%%%%%%%%%%%%%%%%%%%%%%%%%%


\end{document}
